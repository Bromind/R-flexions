\documentclass{article}
\usepackage[utf8]{inputenc}


\title{L'indice 9gag}
\author{Martin \textsc{Vassor}}


\begin{document}
\maketitle
\section{Mesure de concentration}
Autant l'on dispose de bonnes connaissances et de bons outils de mesures d'un ensemble de grandeurs permettant, en somme, de donner l'efficacité\footnote{Au sens général} d'un groupe ou d'un individu à un travail, autantil est difficile de mesurer le degré de concentration d'une personne à la tache qui lui est assignée. La mesure de la concentration est une tâche relativement difficile, car propre à chaque personne.

L'idée proposée est de mesurer le temps de distraction de l'individu. On parle de distraction comme étant une période où l'individu est totalement déconnecté du sujet de travail\footnote{À noter qu'il est important de garder une période de distraction afin de faire une pause dans le travail.}. Un exemple typique est de mesurer le temps passé par la personne sur 9gag\footnote{d'où le nom de l'indice}.
\section{Grandeurs mesurées}
\paragraph{Environnement : }
Étant donné une tâche à faire, le sujet ne doit pas être forcé de travailler, mais encouragé. Il faut lui laisser la possibiliter de gérer son temps, son travail et ses pauses de la manière qu'il le souhaite. Lui interdire explicitement toute pause en dehors de certains horaires prévus conduiraient à maintenir un travail-fantôme.
\paragraph{Le temps entre les distractions : }
Un temps court entre les distraction indique que le sujet n'arrive pas à se reconcentrer. À ce moment, une bonne chose est de prendre le temps de faire une grande pause (voir ci-dessous).
\paragraph{Le temps moyen de distraction par pause : }
Une pause courte indique que le sujet se sent concentré, qu'il a juste besoin de déconnecter quelques minutes entre deux tâches successives. En revanche, une longue distraction indique un état de fatigue avancé. L'avantage de mesurer 9gag est que le sujet n'a pas connaissance à priori de la durée de sa pause. Il ne se connecte pas en se donnant une heure de fin. 
\paragraph{Recoupe des mesures : }
Si le sujet augmente la fréquence de fréquentation de 9gag, avec des temps de plus en plus long, cela indique un besoin de repos. À ce moment, il vaut mieux pour lui se donner un temps suffisamment grand (typiquement au moins le temps de la plus longue session de 9gag), et déconnecter complètement : sortir du lieu de travail, etc.
\section{Utilisation}
L'indice 9gag n'a pas vocation à être utilisé de manière exacte. Il ne se base sur aucune serieuse et en ce sens, constitue plutôt une parodie de différentes mesures étudiées serieusement, qui consistent à théoriser des heuristiques. 
Cependant, il se base sur l'expérience de travail en autonomie que j'ai eu à l'EPFL, sous de fortes contraintes (temps de travail pouvant monter à 100 heures par semaine). Il s'agit plus de donner une aide au travailleur autonôme sur comment détecter la perte de concentration et le besoin de pause.

\end{document}
