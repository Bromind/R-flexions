\documentclass{article}

\usepackage[frenchb]{babel}
\usepackage[utf8]{inputenc}
\usepackage{amsfonts}

\title{Le Libre-arbitre n'est pas le hasard}
\author{Martin \textsc{Vassor}}

\begin{document}
\maketitle

\paragraph{Introduction :}
En attendant la production d'un texte un peu plus précis sur le Libre-Arbitre, 
je vous propose cette petite brève en guise d'introduction.
\subparagraph{}
J'insiste sur le fait que je ne suis absolument pas philosophe ni logicien, et 
qu'il est donc tout à fait possible qu'il existe des enormes erreurs.

\paragraph{Libre-Arbitre vs. Hasard :}
À première vue, il semble que le libre arbitre soit incompatible avec le 
déterminisme. 
\subparagraph{}
Supposons maintenant qu'une personne (Bob) fasse un choix. Alors, deux 
possibilités s'offrent à nous : 
\begin{itemize}
	\item Le choix est déterministe\footnote{On ne s'embête pas avec la 
			question de savoir "Mais si c'est déterministe, alors 
		est-ce que c'est un choix ? etc..."}.
	\item Le choix est issu d'un arbitrage libre.
\end{itemize}

Le but est de montrer que le deuxième cas est contradictoire.

\paragraph{Arbitrage libre :}
Supposons maintenant que nous soyons dotés de la connaissance absolue\footnote{
Wahh... la classe}, et que nous demandons à Bob d'expliquer son choix. Alors 
plusieurs cas sont possibles : 
\begin{itemize}
	\item Bob donne une explication rationnelle qui est la raison effective 
		de son choix.
	\item Bob donne une explication rationnelle, mais il se trompe, soit la 
		raison n'est pas bonne, soit il n'y a pas de raison. 
	\item Bob ne donne pas de raison\footnote{Il peut attribuer son choix à 
		n'importe quoi, le hasard, le libre arbitre, le monstre 
	spaghetti volant, etc...}, mais il existe un raison qu'il ignore.
	\item Bob ne donne pas de raison, et il n'existe pas de raison.
\end{itemize}
\subparagraph{Premier cas : }
Parfaitement déterministe, aucun soucis ;-)
\subparagraph{Second cas : }
Bob donne une raison, il est donc capable d'expliquer rationnellement son choix.
Même si on sait (étant omniscient) qu'il se trompe, il n'a pas conscience 
d'avoir arbitré puisqu'il donne une raison de son choix.\\
La raison effective (ou l'absence de raison) étant indépendante de Bob, soit 
elle est soit déterministe (s'il existe une raison), soit il n'y a pas de raison
et le choix est du au hasard.\\
En tous les cas, Bob n'a rien arbitré.
\subparagraph{Troisième cas : }
Si Bob ne donne pas de raison, il ne peut pas justifier que c'est lui qui a fait
le choix. 
\begin{enumerate}
	\item S'il existe une raison qu'il ignore, pas de soucis, c'est 
		déterministe.
	\item S'il attribue son choix au hasard, alors il ne peut prétendre au 
		libre arbitre, puisque par définition du libre arbitre, le choix
		aurait du venir de lui, et non d'un processus aléatoire.
	\item S'il attribue le choix à toute autre processus, il admet donc 
		qu'un processus non aléatoire ai procédé au choix, donc 
		déterministe.
\end{enumerate}
\subparagraph{Quatrième cas : }
Dans ce cas, le choix est dû au hasard. Bob n'a donc rien arbitré.

\paragraph{Conclusion : }
Donc, après un "article" bien trop long pour l'interet du contenu. Nous mettons 
en évidence le paradoxe suivant : \\
Soit on cherche à déterminiser notre choix, soit on admet que nous n'avons pas 
le controle sur le processus complet du choix. 

Enfin, ce n'est pas parce que nous pouvons accepter l'hypothèse de l'existence 
de processus aléatoires que cela justifie le libre arbitre. Au contraire, 
accepter le hasard est admettre que nous n'avons pas le controle sur notre 
choix.



\end{document}
