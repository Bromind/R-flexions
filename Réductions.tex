\documentclass{article}

\usepackage[frenchb]{babel}
\usepackage[utf8]{inputenc}

\title{$-10\%$ de réduction}
\author{Martin \textsc{Vassor}}

\begin{document}
\maketitle

\paragraph*{}
Je ne suis pas spécialiste des publicités et du marketing en général, mais il 
semble que le principe soit le suivant : les clients sont, à force de subir la 
l'omniprésence des campagnes, formatés pour réagir à certains "signaux". 
Par exemple, en affichant "\textsc{Soldes}", on réagit automatiquement, sans 
avoir à lire le mot. En fait, le cerveau a assimilé le mot "\textsc{Soldes}" 
comme un unique symbole qu'il met en relation avec la notion de solde.
\subparagraph*{}
De la même manière, lorsque nous conduisons et que nous approchons un panneau 
\textsc{stop}, nous ne lisons pas le mot "\textsc{stop}" : le cerveau fait 
automatiquement le lien entre le symbole du panneau et l'action de s'arrêter.

\paragraph*{}
Le problème est que, toujours avide de clients, les campagnes de marketing 
actuelles en arrivent à concatener ces "stimuli", sans doute pour augmenter le 
nombre de symboles perçus. En particulier, depuis quelques temps, j'oberve des 
annonces dans le genre "\textsc{$-10\%$ de réduction}", c'est-à-dire la 
combinaison des symboles "$-10\%$" (sur l'addition par exemple), ou "$-10$ euros"
et du symbole "\textsc{réduction}". Et toute personne non avertie ne relèvera 
pas la différence entre le message effectivement donné et le message perçu.
\subparagraph*{}
En effet, lorsque nous passons rapidement devant une de ces affiches, nous ne 
prenons pas le temps de la lire (et les publicitaires le savent et en 
profitent).
Cependant, même si nous ne la lisons pas, notre cerveau analyse l'image que nous
voyons. Et par le processus expliqué ci-dessus, il nous renvoie automatiquement
une information comme quoi il y aurait une promotion. Bref, malgré nous, et 
malgré que nous n'ayons pas lu l'annonce, nous percevons le message de la 
manière souhaitée par le concepteur de la pub.
\subparagraph*{}
Mais qu'en est-il du message réellement donné ? Et bien il suffit d'ouvrir son 
dictionnaire et son livre de maths de 4\up{ème} et analyser mécaniquement la 
phrase : \\
\begin{center}
"$-10\%$ de réduction"\\
\end{center}
C'est-à-dire, d'après \emph{Larousse}, que le magasin consent à diminuer le prix
de l'objet en question, la valeur de la réduction étant de $-10\%$ du prix 
original.\\
En ouvrant maintenant notre livre de maths, au chapitre sur les pourcentages et 
la proportionnalité, nous pouvons en déduire l'équation suivante : 
$$P_r = P_o - R $$
$P_r$ est le prix après la réduction consentie par le magasin.\\
$P_o$ est le prix original\\
$R$ la valeur de la réduction.\\ 
Or la réduction vaut $-10\%$
comme l'annonce fièrement le publicitaire. On a donc : 
$$R = -10\% \times P_o = -\frac{P_o}{10}$$
Donc, en réintroduisant $R$ dans l'équation initiale :
$$P_r = P_o - R = P_o - \left(-\frac{P_o}{10}\right) = P_o + \frac{P_o}{10}$$
De là, trois possibilités s'offrent à nous : 
\begin{itemize}
	\item[$P_o > 0$]{C'est le cas standard, nous payons une somme positive (
		c'est-à-dire que nous avons moins d'argent après la transaction 
		qu'avant). Dans ce cas le prix soit disant réduit est en fait 
		majoré.}
	\item[$P_o = 0$]{Ce qui arrive lorsqu'on vous offre quelque chose, 
		auquel cas la réduction est nulle.}
	\item[$P_o < 0$]{Dans ce cas, le prix réduit est bien inférieur au prix 
			initial. Il est toutefois dommage que les magasins ne 
		donnent jamais d'argent en plus de leurs produits}
\end{itemize}

\paragraph*{}
Bien sur, en pratique les magasins font bien une réduction de $+10\%$. Mais, en 
il s'agit là bien d'une astuce de publicitaire. 
Cette petite anecdote met en évidence les processus mis en oeuvre par les 
services marketing et le fait que nous y sommes assujettis. 



\end{document}
